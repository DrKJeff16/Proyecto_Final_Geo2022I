% --TODO - Proyecto:
%    - Crear flujo de trabajo []
%    - Asignar temas y trabajos a cada uno []
%    - Crear y configurar el repositorio [X] - Guena

% --TODO - Utilidades de LaTeX:
%    - Averiguar como implementar graficas y demas []
%    - (Opcional) Implementar vista de Presentacion []
%    - (Opcional) Cambiar la fuente del proyecto por una mejor []

\documentclass[letterpaper,14pt]{extreport} % Modo Papel de carta, fuente de 14pt, formato de reporte
% --BUG: Poner 'twosided' (impreso al frente y al reves) arruina el formato

\usepackage[utf8]{inputenc}
\usepackage{amsmath}
\usepackage{amssymb}
\usepackage{ragged2e}
\usepackage{blindtext}
\usepackage{geometry}
\usepackage{enumitem}

% --TODO: - Titulo:
%    - Dar formato al titulo []
%    - Poner correctamente los nombres de autor []
%    - Agregar la materia []
%    - Agregar la fecha oficial []

\title{Cambios de Coordenadas Rectangulares a Esféricas} % El titulo del documento
\author{Guennadi Maximov Cortés, Johan Wences} % Pon tu nombre completo :v
\date{19/01/2022} % Cambiar la fecha cada vez que se edite este documento

% --FIXME: Eliminar la numeracion tipo "0.X.X" de la Tabla de Contenidos
\renewcommand*\contentsname{Índice} % Da formato a la Tabla de Contenidos

% -- Elimina el texto "Chapter N"
\makeatletter
\def\@makechapterhead#1{
  \vspace*{50\p@}{\parindent\z@ \raggedright\normalfont
  \ifnum\c@secnumdepth>\m@ne
    \if @mainmatter\Huge\bfseries\space\thechapter.\space
    \fi
  \fi
    \interlinepenalty\@M\Huge\bfseries#1\par\nobreak\vskip 40\p@
  }}
\makeatother


% --TODO: - Documento:
%    - Crear las secciones [X] - Guena
%    - Crear y formatear el texto []

\begin{document}

  \maketitle
  \pagenumbering{arabic}
  \tableofcontents
  \newpage
  
  \chapter{Introducción}
    En este documento se explicarán las coordenadas esféricas, sus propiedades, sus aplicaciones y cómo convertirlas a otros sistemas de coordenadas.
  \section{Marco histórico}
  \section{Abstracto}
  
  \chapter{Desarrollo}
  \renewcommand{\chaptername}{Jornada}
  \section{Introducción}
    \subsection{¿Qué es una esfera?}
    \subsection{Ecuación de una esfera}
    \subsection{¿Qué son las coordenadas esféricas?}
    \subsection{Gráfica de un punto en coordenadas esféricas}
  \section{Conversión de coordenadas}
    \subsection{De coordenadas rectangulares a esféricas}
    \subsection{De coordenadas esféricas a rectangulares}
  \section{Conclusiones}
  
  \chapter{Referencias}
  
\end{document}
