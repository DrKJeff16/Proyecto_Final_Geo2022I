% --TODO - Proyecto:
%    - Crear flujo de trabajo []
%    - Asignar temas y trabajos a cada uno []
%    - Crear y configurar el repositorio [X] - Guena

% --TODO - Utilidades de LaTeX:
%    - Averiguar como implementar graficas y demas []
%    - (Opcional) Implementar vista de Presentacion []
%    - (Opcional) Cambiar la fuente del proyecto por una mejor []

\documentclass[letterpaper,14pt]{extreport} % Modo Papel de carta, fuente de 14pt, formato de reporte
% --BUG: Poner 'twosided' (impreso al frente y al reves) arruina el formato

\usepackage[utf8]{inputenc}
\usepackage{amsmath}
\usepackage{amssymb}
\usepackage{ragged2e}
\usepackage{blindtext}
\usepackage{geometry}
\usepackage{enumitem}

% --TODO - Titulo:
%    - Dar formato al titulo []
%    - Poner correctamente los nombres de autor []
%    - Agregar la materia []
%    - Agregar la fecha oficial []

\title{Cambios de Coordenadas Rectangulares a Esféricas} % El titulo del documento
\author{Guennadi Maximov Cortés, Johan Wences} % Pon tu nombre completo :v
\date{11/01/2022} % Cambiar la fecha cada vez que se edite este documento

% --FIXME: Eliminar la numeracion tipo "0.X.X" de la Tabla de Contenidos
\renewcommand*\contentsname{Índice} % Da formato a la Tabla de Contenidos

% --TODO - Documento:
%    - Crear las secciones [X] - Guena
%    - Crear y formatear el texto []

\begin{document}
  \maketitle
  \pagenumbering{arabic}
  \tableofcontents
  \newpage

  \section{Introducción}
    \subsection{Marco Histórico}
    \subsection{Abstracto}
  \section{Desarrollo}
    \subsection{Introducción}
      \subsubsection{¿Qué es una esfera?}
      \subsubsection{Ecuación de una esfera}
    \subsection{Conversión de coordenadas}
       \subsubsection{Conversión de coordenadas rectangulares a esféricas}
       \subsubsection{Conversión de coordenadas esféricas a rectangulares}
    \subsection{Conclusiones}
      \subsubsection{Ejercicios de aplicación}
  \section{Bibliografía y Referencias}
\end{document}
