\hspace{4mm} El uso de dichos sistemas de coordenadas en amplios campos del conocimiento refleja la utilidad del estudio de la localización de puntos en superficies esféricas. En cuando al uso de los distintos sistemas dependerá de la solución esperada en el desarrollo del problema que se esté tratando en particular. Tan sólo en la conversión de los distintos sistemas se encuentran ciertas dificultades en cada una de ellas. Por ejemplo, para convertir coordenadas de rectangulares a esféricas, en la obtención del ángulo ${\theta}$ se encuentra una ligera dificultad porque se necesita conocer la ubicación del punto y en función de éste encontrar el ángulo auxiliar más cercano al eje x que permita complementar la búsqueda del ángulo ${\theta}$. En cambio utilizando coordenadas esféricas resulta trivial conocer los valores de sus equivalencias respecto al sistema rectangular. \cite{ecured}