La ecuación de la esfera se deriva de la ecuación de la circunferencia, a partir de su forma más simple:

\begin{eqnarray*}
  x^2+y^2=r^2
\end{eqnarray*}

Agregando un nuevo eje ${z}$ y, a su vez, agregando una coordenada ${z}$ a los puntos de la circunferencia, se crea un triángulo rectángulo imaginario conteniendo como catetos el radio de la circunferencia original y el valor de la coordenada ${z}$:

\begin{eqnarray*}
  r^2+z^2=R^2
\end{eqnarray*}

Si sustituimos el radio de la circunferencia por sus valores ${x}$ y ${y}$, tenemos:

\begin{eqnarray*}
  x^2+y^2+z^2=R^2
\end{eqnarray*}

En caso de un centro variable con coordenadas ${\left(h,k,l\right)}$, si aplicamos una transformación de traslación, tenemos la ecuación general de una esfera:

\begin{eqnarray*}
  \left(x-h\right)^2+\left(y-k\right)^2+\left(z-l\right)^2=R^2
\end{eqnarray*}

Si desarrollamos la ecuación anterior, llegamos a la forma general de una esfera:

\begin{eqnarray*}
  x^2+y^2+z^2+Ax+By+Cz+D=0
\end{eqnarray*}
