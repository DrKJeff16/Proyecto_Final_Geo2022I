Para localizar un punto en un plano se necesitan dos números. Se sabe que cualquier punto en el plano se puede representar como un par ordenado ${(a,b)}$ de números reales, donde ${a}$ es la coordenada ${x}$ y ${b}$ es la coordenada ${y}$. Por ésta razón se dice que un plano es bidimensional. Para localizar un punto en el espacio se requieren tres números. Se representa cualquier punto en el espacio mediante una terna ordenada ${(a, b, c)}$ de números reales, al que se le conoce como espacio tridimensional.

\vspace{4mm}
\textbf{Superficies:} En geometría analítica bidimensional, la gráfica de una ecuación que implica ${x}$ y ${y}$ es una curva en ${\mathbb{R}^{2}}$. En geometríá analítica tridimensional, una ecuación en ${x}$, ${y}$ y ${z}$ representa una superficie en ${\mathbb{R}^{3}}$. 

% Parrafos extraído del libro "Cálculo: trascendentes tempranas, James Stewart"
\label{espacio_tridimensional}
