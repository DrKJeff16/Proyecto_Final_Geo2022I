Mediante la siguiente ilustración se observa el punto rojo en el espacio tridimensional con coordenadas esféricas $(\rho,\theta,\varphi)$, que proyecta un plano perpendicular al plano $(x,y)$, en dicho espacio se forman $2$ triángulos rectángulos. Es importante señalar que la variable $r$ es una coordenada polar que ayuda en la formación del triángulo rectángulo y su relación con las coordenadas $(x,y)$. Por ello, para el ángulo interno $\varphi$ se calcularán sus funciones trigonométricas.

\begin{figure}[H]
  \centering
  \includegraphics[width=11.17cm, height=5.67cm]{img/graph/coord_esf_2_rect_1.jpg}
  \caption{Triángulo rectángulo de un plano tridimensional.}
  \label{relacion_de_coordenadas}
\end{figure}

Entonces,
\[cos \varphi = \frac{\text{cateto adyacente}}{\text{hipotenusa}} = \frac{z}{\rho} \rightarrow z = \rho cos \varphi\]
\[sin \varphi = \frac{\text{cateto opuesto}}{\text{hipotenusa}} = \frac{r}{\rho} \rightarrow r = \rho sin \varphi\]

\vspace{4mm}
Pero por la relación que se estableció con las coordenadas rectangulares y polares, entonces:
\[x = r cos \theta \rightarrow x = \rho sin \varphi cos \theta = \rho cos \theta sin \varphi\]
\[y = r sin \theta \rightarrow y = \rho \sin \varphi sin \theta = \rho sin \theta sin \varphi\]

\vspace{4mm}
Ahora, cuando se tienen coordenadas esféricas con el uso de dichas ecuaciones, se podrá determinar en el sistema de coordenadas $(x,y,z)$ (rectangulares).
